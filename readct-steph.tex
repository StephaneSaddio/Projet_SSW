% Options for packages loaded elsewhere
\PassOptionsToPackage{unicode}{hyperref}
\PassOptionsToPackage{hyphens}{url}
%
\documentclass[
]{article}
\usepackage{amsmath,amssymb}
\usepackage{lmodern}
\usepackage{iftex}
\ifPDFTeX
  \usepackage[T1]{fontenc}
  \usepackage[utf8]{inputenc}
  \usepackage{textcomp} % provide euro and other symbols
\else % if luatex or xetex
  \usepackage{unicode-math}
  \defaultfontfeatures{Scale=MatchLowercase}
  \defaultfontfeatures[\rmfamily]{Ligatures=TeX,Scale=1}
\fi
% Use upquote if available, for straight quotes in verbatim environments
\IfFileExists{upquote.sty}{\usepackage{upquote}}{}
\IfFileExists{microtype.sty}{% use microtype if available
  \usepackage[]{microtype}
  \UseMicrotypeSet[protrusion]{basicmath} % disable protrusion for tt fonts
}{}
\makeatletter
\@ifundefined{KOMAClassName}{% if non-KOMA class
  \IfFileExists{parskip.sty}{%
    \usepackage{parskip}
  }{% else
    \setlength{\parindent}{0pt}
    \setlength{\parskip}{6pt plus 2pt minus 1pt}}
}{% if KOMA class
  \KOMAoptions{parskip=half}}
\makeatother
\usepackage{xcolor}
\IfFileExists{xurl.sty}{\usepackage{xurl}}{} % add URL line breaks if available
\IfFileExists{bookmark.sty}{\usepackage{bookmark}}{\usepackage{hyperref}}
\hypersetup{
  pdftitle={Rapport du projet de M1},
  pdfauthor={Wiam Chaoui; Sophie Manuel; Stéphane Sadio},
  hidelinks,
  pdfcreator={LaTeX via pandoc}}
\urlstyle{same} % disable monospaced font for URLs
\usepackage[margin=1in]{geometry}
\usepackage{graphicx}
\makeatletter
\def\maxwidth{\ifdim\Gin@nat@width>\linewidth\linewidth\else\Gin@nat@width\fi}
\def\maxheight{\ifdim\Gin@nat@height>\textheight\textheight\else\Gin@nat@height\fi}
\makeatother
% Scale images if necessary, so that they will not overflow the page
% margins by default, and it is still possible to overwrite the defaults
% using explicit options in \includegraphics[width, height, ...]{}
\setkeys{Gin}{width=\maxwidth,height=\maxheight,keepaspectratio}
% Set default figure placement to htbp
\makeatletter
\def\fps@figure{htbp}
\makeatother
\setlength{\emergencystretch}{3em} % prevent overfull lines
\providecommand{\tightlist}{%
  \setlength{\itemsep}{0pt}\setlength{\parskip}{0pt}}
\setcounter{secnumdepth}{5}
\usepackage{fancyhdr}
\pagestyle{fancy}
\fancyhead[L]{Combient de temps pour crée une espèce ?}
\fancyhead[R]{\emph{M1 - Biostatistique}}
\usepackage{dsfont}
\usepackage{subfig}
\usepackage{amsmath}
\usepackage{amssymb}
\usepackage{amsthm}
\usepackage{amsmath}
\usepackage{dsfont}
\ifLuaTeX
  \usepackage{selnolig}  % disable illegal ligatures
\fi

\title{Rapport du projet de M1}
\usepackage{etoolbox}
\makeatletter
\providecommand{\subtitle}[1]{% add subtitle to \maketitle
  \apptocmd{\@title}{\par {\large #1 \par}}{}{}
}
\makeatother
\subtitle{Combien de temps pour faire une espèce ?}
\author{Wiam Chaoui \and Sophie Manuel \and Stéphane Sadio}
\date{2021}

\begin{document}
\maketitle

\newthm{dfn}{Définition}
\newthm{exemple}{Exemple}
\newthm{corollary}{Corollaire}
\newthm*{prop}{prop}
\newthm{lemma}{Lemme}
\newthm*{demo}{Démonstration}
\newthm{rem}{Remarque}
\newthm{propetie}{Propriété}
\newthm{nb}{NB}
\newthm{thm}{Théorème}
\newpage
\tableofcontents
\newpage

\hypertarget{risque-quadratique-des-estimateurs-uxe0-noyau-sur-les-classe-des-espaces-de-huxf6lder}{%
\subsubsection{Risque quadratique des estimateurs à noyau sur les classe
des espaces de
Hölder}\label{risque-quadratique-des-estimateurs-uxe0-noyau-sur-les-classe-des-espaces-de-huxf6lder}}

Nous nous intéressons au risque quadratique de \(\hat{f}_n\), définit
par :\\
étant donné \(x_0 \in \mathbb{R}\) \[
R(\hat {f}_n, f) = \mathbb{E}[|\hat {f}_n(x_0) - f(x_0)|^2]
\]

Rappelons la décomposition ``biais-variance'' du risque quadratique : \[
\mathbb{E}[|\hat {f}_n(x_0) - f(x_0)|^2] = (\mathbb{E}[\hat {f}_n(x_0)] - f(x_0))^2 + \mathbb{V}[\hat {f}_n(x_0)]
\]

\hypertarget{majoration-du-biais-et-de-la-variance}{%
\paragraph{Majoration du biais et de la
variance}\label{majoration-du-biais-et-de-la-variance}}

Dans cette section, nous allons nous intéresser au compromis
biais-variance afin de minimiser le risque quadratique. Nous
introduirons après quelques définitions deux propositions qui montrent
que sous certaines hypothèses, on peut majorer le biais ainsi que la
variance.\newline

\begin{dfn} Pour tout $\beta > 0$ et $L > 0$, on définit la classe de Hölder de régularité $\beta$ et de rayon $L$ par

$$
  \Sigma(\beta,L)={\{f:\mathbb{R} \longrightarrow \mathbb{R}\ \text{t.q.}\ f\ \text{est} \left\lfloor{\beta}\right\rfloor\ \text{fois dérivable et}   \\
  \forall\ (x,y) \in \mathbb{R}_2\ \ \mid f^{\left\lfloor{\beta}\right\rfloor}(y)-f^{\left\lfloor{\beta}\right\rfloor}(x)\mid \leq L{\mid x-y\mid}^{\beta - \left\lfloor{\beta}\right\rfloor}\}}
$$

On notera $\Sigma_d(\beta,L)$ l'intersection de $\Sigma(\beta,L)$ et l'ensemble des densités.

\end{dfn}
\begin{rem} _ Si $\beta = 1$ on obtient l'ensemble des fonctions $L$-lipschitziennes.\newline
_ Si $\beta > 1$ alors $f'\in \Sigma(\beta-1,L)$.
\end{rem}
\begin{prop} (admise) Soit $\beta > 0$ et $L > 0$, il existe une constante $M(\beta, L)$ telle que

$$
\underset{f \in \Sigma_d(\beta,L)}{sup}{\parallel f \parallel}_{\infty}= \underset{x \in \mathbb{R}}{sup}\ \underset{f \in \Sigma_d(\beta,L)}{sup}f(x) \leq M(\beta,L)
$$
\end{prop}

\begin{dfn} Soit $\ell \in \mathbb{N^*}$. On dit que le noyau $K$ est d'ordre $\ell$ si $u^jK(u)$ est intégrable et 
$\int u^jK(u)du =  0,\   \ j = {1,...,\ell}$

\end{dfn}
\begin{prop}: Si $f \in \Sigma(\beta,L)$ avec $\beta > 0$ et $L > 0$ et si $K$ est un noyau d'ordre $\ell = \left\lfloor{\beta}\right\rfloor$ tel que $\int |{u}^{\beta}|\,.|{K(u)}|~du < \infty$ alors pour tout $x_0 \in \mathbb{R}$, et pour tout $h>0$ le biais peut être borné comme suit:

$$
|\mathbb{E}[\hat{f}_n(x_0)] - f(x_0)|\leqslant \frac{h^{\beta}L}{\ell!}\int|u|^{\beta}|K(u)|du
$$
\end{prop}

\begin{demo} On a
$$
\begin{aligned}
\mathbb E\lgroup \hat f_n(x_0) \rgroup&=\mathbb E\lgroup\frac{1}{n}\sum_{i=1}^n \frac{1}{h}K(\frac{X_i-x_0}{h})\rgroup \\
&=\mathbb E\lgroup \frac{1}{h}K(\frac{X_1-x_0}{h})\rgroup \\
&=\frac{1}{h}\int K(\frac{u-x_0}{h})f(u)du  \\
&=\int K(v)f(x_0+hv)dv,\  (en\ posant\ v=\frac{u-x_0}{h}).
\end{aligned}
$$
De plus
$$
f(x_0)=f(x_0)\times 1 = f(x_0) \int K(v)dv.
$$
Comme $f \in \Sigma(\beta, L)$, $f$ admet $\left\lfloor{\beta}\right\rfloor}$ dérivées et par un développement de Taylor-Lagrange on a, pour tout $x \in \mathbb R$,

$$
f(x)= \sum_{i=1}^{\ell-1}\frac{(x-x_0)^k}{k!}f^{(k)}(x_0)+\frac{(x-x_0)^\ell}{\ell!}f^{(\ell)}(x_0+\zeta (x-x_0))
$$  
avec $\zeta \in ]0,1[$. Autrement dit on a, avec $x= x_0+hv$,
$$
f(x_0+hv)-f(x_0)=\sum_{i=1}^{\ell-1}\frac{(hv)^k}{k!}f^{(k)}(x_0)+f^{(\ell)}(x_0+hv\zeta)\frac{(hv)}{\ell !}
$$  
pour un certain $\zeta \in ]0,1[$. Donc

$$
\begin{aligned}
\int K(v)\lgroup f(x_0+hv)-f(x_0)\rgroup dv &= \int K(v)\lgroup\sum_{i=1}^{\ell - 1}f^{(k)}(x_0)+f^{(\ell)}(x_0+hv\zeta)\frac{(hv)^{\ell}}{\ell !}\rgroup dv \\
&=\frac{h^{\ell}}{\ell !}\int K(v)v^{\ell}f^{(\ell)}(x_0+hv\zeta)dv
\end{aligned}
$$

Comme $K$ est d'ordre $\ell$, on a aussi $\int K(v)v^{\ell}f'^(\ell)}(x_0)dv=0$. Donc on a
$$
\int K(v)\lgroup f(x_0+hv)-f(x_0)\rgroup dv = \frac{h^{\ell}}{\ell !}\int K(v)v^{(\ell)}\lgroup f^{(\ell)}(x_0 + hv\zeta)-f^{(\ell)}(x_0)\rgroup dv
$$
Or, $f \in \Sigma(\beta, L)$, on a donc $\mid f^{(\ell)}(x_0+hv\zeta)-f^{(\ell)}(x_0)\mid \leq L|hv|^{\beta-\ell}$. Et finalement
$$
\mid \int K(v)\lgroup f(x_0+hv\zeta)-f(x_0)\rgroup dv \mid \leq \frac{\mid h\mid^{\beta}}{\ell !}\int \mid K(v) \mid |v|^{\ell}l|hv|^{\beta-\ell}dv
$$

ce qui signifie que
$$
|\mathbb E \lgroup\hat f_n(x_0)-f(x_0)| \leq \frac{L|h|^{\beta}}{\ell !}\int |K(v)||v|^{\beta}dv
$$
\end{demo}
\begin{corollary}   Le biais au carré tend vers zéro à la vitesse $h^{2\beta}$. Plus la fonction $f$ est régulière, plus le biais tend vite vers zéro quand $h$ tend vers zéro (à condition bien sûr que l'ordre du noyau soit suffisamment grand). Nous en déduisons la convergence de l'espérance de l'estimateur à noyau $\hat {f}_n$ vers la fonction $f$. Et donc, l'estimateur à noyau est asymptotiquement sans biais, $\hat {f}_n$ est donc consistant.\newline
 \end{corollary}
\begin{prop}: Si $f$ est bornée et si $K$ est de carré intégrable alors 

$$
\mathbb{V}(\hat {f}_n(x_0)) \leqslant \frac{\begin{Vmatrix}f\end{Vmatrix}_{\infty}\begin{Vmatrix}K\end{Vmatrix}^2_2}{nh}
$$

En particulier, si $f \in \Sigma(\beta,L)$ alors

$$
\mathbb{V}(\hat {f}_n(x_0)) \leqslant \frac{M(\beta, L)\begin{Vmatrix}K\end{Vmatrix}^2_2}{nh}
$$

\end{prop}
\begin{demo}:
$$
\begin{aligned}
\mathbb{V}(\hat {f}_n(x_0)) &= \mathbb{V}(\frac{1}{nh}\sum_{i=1}^nK(\frac{X_i-x_0}{h})) \\
&=\sum_{i=1}^n\mathbb{V}(\frac{1}{nh}K(\frac{X_i-x_0}{h})) \\
&=\sum_{i=1}^n\frac{1}{n^2h^2}\mathbb{V}(K(\frac{X_i-x_0}{h})) \\
&=\frac{1}{nh^2}\mathbb{V}(K(\frac{X_1-x_0}{h}) \\
&\leqslant \frac{1}{nh^2}\mathbb{E}(K^2(\frac{X_1-x_0}{h})) \\
&=\frac{1}{nh^2}\int K^2(\frac{u-x_0}{h})f(u)du \\
&=\frac{1}{nh}\int K^2(v)f(x_0 +vh)dv
\end{aligned}
$$ 

Et enfin,on utilise la proposition ? : il existe une constante positive $M(\beta,L)$ tel que $\begin{Vmatrix}f\end{Vmatrix}_{\infty} \leqslant M(\beta, L)$. Ceci implique que :
$$
 \mathbb{V}(\hat {f}_n(x_0))\leqslant\frac{1}{nh}M(\beta, L)\int K^2(v)dv 
$$ 
 \end{demo}

Pour que la variance tende vers zéro, il faut que \(nh\) tende vers
l'infini. En particulier, à \(n\) fixé, la variance est une fonction
décroissante de \(h\). Il y a donc une valeur optimale de \(h\) qui doit
réaliser l'équilibre entre le biais au carré et la variance. On peut à
présent donner un contrôle du risque quadratique par le théorème
suivant.

\begin{thm} Soit $\beta>0$ et $L>0$ et $K$ un noyau de carré intégrable et d'ordre $\left\lfloor{\beta}\right\rfloor$ tel que $\int |u^{\beta}|\,.|K(u)|du<\infty$. Alors, en choisissant une fenêtre de la forme $h=cn^{-\frac{1}{2\beta+1}}$ avec une constante $c>0$, on obtient pour tout $x_0 \in \mathbb{R}$,

$$ 
R(\hat {f}_n(x_0)),\Sigma_d(\beta, L)):= \underset{f\in\Sigma_d(\beta,L)}{sup}\mathbb{E}[|\hat {f}_n(x_0)-f(x_0)|^2]\leqslant Cn^{-\frac{2\beta}{2\beta+1}}
$$ 
 où $C$ est une constante dépendant de $L,~\beta,~ c$ et $K$.
 \end{thm}
\begin{demo}: 
  On a :
$$
 R(\hat {f}_n(x_0),f(x_0))= \text{Biais au carré + Variance}
$$ 

   Si nous nous référons aux deux propositions précédentes, nous pouvons écrire :

$$
 R(\hat {f}_n(x_0),f(x_0))\leqslant(\frac{h^{\beta}L}{l!}\int |u|^{\beta}|K(u)|du)^2 + \frac{M(\beta,L)\begin{Vmatrix}K\end{Vmatrix}_2^2}{nh}
$$

On cherche ensuite la fenêtre $h$ qui minimise cette quantité. Comme on ne se soucie pas vraiment des constantes exactes quand on cherche la vitesse de convergence d'un estimateur, on utilisera la notation $c_1=(\frac{L}{l!}\int |u|^{\beta}|K(u)|du)^2$ et $c_2=\frac{M(\beta,L)\begin{Vmatrix}K\end{Vmatrix}_2^2}{nh}$. On doit alors minimiser en $h$ la quantité :
$$
  c_1h^{2\beta}+\frac{c_2}{nh}
$$

On a une somme d'une quantité croissante et une quantité décroissante en $h$. Encore une fois, comme on ne se soucie pas pas des constantes, donc on cherche la fenêtre $h$ qui nous donne l'ordre minimal du risque. Quand $h$ est trop grand, le biais est trop grand, et quand $h$ est trop petit, c'est la variance qui est trop grande (voir ...). On cherche donc la fenêtre $h$ qui réalise un équilibre entre le biais au carré et la variance:

$$ 
  h^{2\beta}\approx\frac{1}{nh}
$$
où le signe $\approx$ signifie ici "de l'ordre de". Cela donne :

$$
  h\approx n^{-\frac{1}{2\beta +1}}
$$

Autrement dit, pour une fenêtre $h$ de l'ordre de $n^{-\frac{1}{2\beta+1}}$, le biais au carré et la variance sont de même ordre.Plus exactement, on choisit la fenêtre $h_*=cn^{-\frac{1}{2\beta+1}}$, avec $c$ une constante positive, on a :
$$
  \text{Biais au carré} \approx h_{*}^{2\beta}\approx \text{Variance} \approx \frac{1}{nh_{*}}
$$

De plus, on a alors :
$$
  h_* \approx n^{-\frac{2\beta}{2\beta + 1}}
$$

Autrement dit, il existe une certaine constante $C$ telle que, pour cette fenêtre $h_*$, on a :
$$
  R(\hat {f}_n(x_0),\sum_d(\beta,L))\leqslant Cn^{\frac{-2\beta}{2\beta + 1}}
$$

  Cette fenêtre est donc optimale à une constante près (si on change $c$, on change $C$ ça ne change pas le taux qui est $n^{\frac{-2\beta}{2\beta+1}}$).\newline
\end{demo}
\begin{rem}: L'estimateur dépend de $\beta$ à travers la fenêtre $h$. Or, sans   connaissance a priori sur les propriétés de la fonction $f$, on ne peut donc pas utiliser cet estimateur. On essaie alors de trouver un choix de fenêtre ne dépendant que des données et qui soit aussi performant (ou presque) que l'estimateur utilisant cette fenêtre optimale. A ce sujet, on introduira plus loin un choix de fenêtre ne dépendant que des données et qui est basé sur ce qu'on appelle la validation croisée (ou "cross validation" en Anglais).  \newline
\end{rem}

\end{document}
